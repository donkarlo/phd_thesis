\chapter{Learning in Robots}
Learning refers to the process of acquiring or updating knowledge, skills, or behavior based on data, experience, or instruction. It involves adapting internal models to better represent the external world or to improve task performance. Robots use learning algorithms like supervised, unsupervised, or reinforcement learning to build capabilities such as object recognition or path planning. For example, a robot might learn to classify objects in its environment by training on labeled images. Learning provides the foundational knowledge needed for reasoning and decision-making. \cite{lecun-2015-deep-learning}

\section{Learning Types in Robotics}

\subsection{Supervised Learning}
Supervised learning is a method where a model learns to map inputs to outputs using labeled datasets. It is extensively used for tasks requiring high accuracy, such as object detection and signal classification. In drones, supervised learning can be employed to classify objects in the environment using camera data. For instance, a drone can be trained to recognize trees, vehicles, and humans by using a dataset of labeled aerial images. This capability is crucial for autonomous navigation and obstacle avoidance in complex terrains. \cite{lecun-2015-deep-learning}

\subsection{Unsupervised Learning}
Unsupervised learning discovers hidden patterns or intrinsic structures in unlabeled data, making it useful for exploring unknown environments. In drones, unsupervised learning can process data from LiDAR and IMU sensors to cluster 3D point clouds into meaningful groups, such as open paths, vegetation, or obstacles. This allows the drone to map and understand uncharted areas without explicit labeling, enabling autonomous navigation in new environments. \cite{hinton-1999-unsupervised-learning}

\subsection{Semi-Supervised Learning}
Semi-supervised learning combines labeled and unlabeled data to improve model efficiency in data-scarce scenarios. For a drone navigating a forest, a small amount of labeled LiDAR data can help infer the characteristics of a larger, unlabeled dataset, enabling it to identify safe paths or landing zones. This approach bridges the gap between fully supervised and unsupervised learning in dynamic environments. \cite{zhu-2009-semi-supervised-learning}

\subsection{Multi Modal learning}


\subsection{Cross modal learning}

\subsection{Transfer Learning}
Transfer learning reuses knowledge from a pre-trained model to improve performance on a related task, reducing the need for extensive training. In drones, a model pre-trained to detect vehicles in urban environments using camera data can be fine-tuned to recognize animals in rural areas. This adaptability allows drones to operate effectively across diverse missions with minimal retraining. \cite{pan-2010-transfer-learning}

\subsection{Self-Supervised Learning}
Self-supervised learning generates labels from data itself, enabling the model to learn meaningful representations without manual annotations. A drone can use self-supervised learning to predict future LiDAR readings based on its current IMU and GPS data, allowing it to understand motion dynamics and adapt to changing environmental conditions autonomously. \cite{chen-2020-self-supervised}

\subsection{Meta-Learning (Learning to Learn)}
Meta-learning, or "learning to learn," equips models with the ability to quickly adapt to new tasks using minimal data. For example, a drone trained to navigate deserts can adapt to forested environments with only a few examples by leveraging its prior experience with obstacle avoidance and pathfinding. This rapid adaptability is essential for multi-environment missions. \cite{finn-2017-meta-learning}

\subsection{Continual Learning}
Continual learning allows drones to acquire new skills incrementally while retaining previously learned knowledge, addressing the problem of catastrophic forgetting. For instance, a drone initially trained to navigate open fields can incrementally learn to operate in dense urban areas without forgetting its prior navigation skills. This capability is vital for long-term autonomous operation in dynamic settings. \cite{parisi-2019-continual-learning}

\subsection{Reinforcement Learning}
Reinforcement learning trains agents to make sequential decisions by maximizing cumulative rewards. In drones, reinforcement learning can optimize flight paths by balancing energy consumption and coverage. For instance, a drone might learn to map a large area efficiently by receiving rewards for increased coverage and penalties for excessive battery usage, refining its navigation strategy through trial and error. \cite{sutton-2018-reinforcement-learning}

\textbf{Questions}
\\
Are rewards also learnable? For example, if I have a drone that sends random control inputs to its rotators in a jungle, can it understand control inputs are good that help it increase it power source or balance?

\subsubsection{Q-learning}
Q-learning is a reinforcement learning algorithm that enables an agent to find optimal actions in various states without needing a model of the environment. A drone navigating a maze of obstacles can use Q-learning to identify the best path to its destination by updating a Q-table based on LiDAR-based proximity sensors and rewards for collision-free navigation.

\subsection{Active Learning}
Active learning minimizes the effort required for labeling data by focusing on the most informative samples. For example, a drone might identify ambiguous objects in its camera feed, such as unusual terrain features, and request human input for labeling. This targeted approach improves model accuracy while reducing annotation costs, making it ideal for exploration tasks. \cite{settles-2009-active-learning}

\subsection{Imitation Learning}
Imitation learning enables robots to mimic expert behaviors by observing demonstrations. For instance, a drone can learn to navigate through narrow corridors by analyzing the control inputs and camera footage of an experienced pilot. This approach reduces the need for explicit programming in scenarios requiring complex decision-making. \cite{argall-2009-imitation-learning}

\subsection{Online Learning}
Online learning allows models to update incrementally as new data becomes available, ensuring adaptability to dynamic environments. A delivery drone, for example, might continuously refine its flight model based on real-time wind data from its IMU and GPS sensors, maintaining stability and improving efficiency in varying conditions. \cite{hoi-2018-online-learning}

\subsection{Evolutionary Learning}
Evolutionary learning optimizes models through algorithms inspired by natural selection, such as mutation and crossover. A swarm of drones might use evolutionary learning to evolve collaborative strategies for mapping large areas, where each generation improves performance by combining the strengths of previous attempts. \cite{goldberg-1989-genetic-algorithms}

\subsection{Adversarial Learning}
Adversarial learning involves training models in adversarial setups to improve robustness and generalization. For example, a drone can learn to detect and reject spoofed GPS signals by training with adversarially generated GPS noise, ensuring reliable operation even in challenging environments. \cite{goodfellow-2014-gan}

\subsection{Federated Learning}
Federated learning enables collaborative model training across multiple drones without sharing raw data, preserving privacy and security. Each drone trains a local model using its camera and LiDAR data, and the aggregated updates improve a global model shared across the fleet. This approach is particularly useful in sensitive applications such as surveillance or disaster response. \cite{mcmahan-2017-federated-learning}

\subsection{Curriculum Learning}
Curriculum learning organizes training by progressing from simpler to more complex tasks, mimicking human learning. For example, a drone may first learn to hover, then navigate open spaces, and finally handle complex, obstacle-rich environments like dense forests. This structured approach improves learning efficiency and robustness. \cite{bengio-2009-curriculum-learning}

\subsection{Multi-Task Learning}
Multi-task learning enables models to perform several tasks simultaneously by leveraging shared representations. For instance, a drone might simultaneously detect obstacles using LiDAR and estimate wind conditions with IMU data, optimizing its operations for both safety and efficiency. \cite{caruana-1997-multi-task-learning}

\subsection{Intrinsic Motivation Learning}
Intrinsic motivation learning encourages agents to explore autonomously by leveraging curiosity or novelty. A drone might prioritize exploring areas with high variability in terrain, using LiDAR and GPS data to discover new landmarks, driven by its intrinsic desire to gather information about its environment. \cite{oudeyer-2007-intrinsic-motivation}

\subsection{Zero-Shot Learning}
Zero-shot learning enables models to generalize to unseen classes or tasks using semantic relationships or prior knowledge. For example, a drone trained to identify common tree species can recognize a new species based on its similarity to known categories, using visual and structural features from its camera and LiDAR data. \cite{larochelle-2008-zero-shot-learning}

\subsection{Few-Shot Learning}
Few-shot learning allows models to learn tasks with minimal labeled examples, leveraging prior knowledge and generalization. A search-and-rescue drone could quickly identify specific types of debris or hazards in disaster areas using only a few labeled images, enhancing its response capabilities in emergencies. \cite{vinyals-2016-few-shot-learning}

\subsection{sensory motor learning}

\section{Reasoning}
Reasoning is the cognitive process of deriving conclusions, solving problems, or making decisions based on existing knowledge or logic. Unlike learning, reasoning often operates on pre-existing knowledge or models to produce actionable insights. It encompasses techniques like deductive reasoning (general to specific), inductive reasoning (specific to general), and abductive reasoning (best plausible explanation). In robotics, reasoning allows systems to decide the best course of action, such as selecting a path to avoid an obstacle. Reasoning leverages learned information to achieve goals efficiently. \cite{russell-2010-artificial-intelligence}

\section{Inference}
Inference is the act of drawing specific conclusions or making deductions based on evidence or observed data. It can involve probabilistic reasoning, such as Bayesian inference, where a robot estimates the likelihood of events based on prior knowledge and current observations. For example, a drone might infer that a detected object is a tree based on sensory data from its camera and LiDAR. Inference often operates as a step within broader reasoning processes, making it a more targeted, evidence-driven activity. \cite{bishop-2006-pattern-recognition}

Inference techniques in robotics enable robots to derive conclusions, make decisions, or interpret data based on sensory inputs, learned models, and pre-existing knowledge. These techniques handle uncertainties, optimize actions, and allow robots to adapt to dynamic environments. Below, we discuss key inference methods used in robotics.



\subsection{Comparison Table}
\begin{table}[h!]
    \centering
    \begin{tabular}{|l|l|l|l|}
        \hline
        \textbf{Aspect} & \textbf{Learning} & \textbf{Reasoning} & \textbf{Inference} \\ \hline
        \textbf{Definition} & Acquiring knowledge or skills through experience or data. & Deriving conclusions or solving problems based on knowledge or logic. & Drawing specific conclusions from data or observations. \\ \hline
        \textbf{Input} & External data or experiences. & Pre-existing knowledge or models. & Evidence or current observations. \\ \hline
        \textbf{Output} & Updated knowledge or internal model. & Decisions, plans, or broader insights. & Specific, evidence-based conclusions. \\ \hline
        \textbf{Scope} & Broad, involves building general understanding. & Broader problem-solving or decision-making process. & Narrow, focused on individual conclusions. \\ \hline
        \textbf{Examples in Robots} & Training a model to recognize objects in the environment. & Deciding the best route to avoid obstacles based on a map. & Inferring that a detected object is a tree using sensory data. \\ \hline
        \textbf{Relation} & Provides the foundation for reasoning. & Uses knowledge from learning to guide inference and decision-making. & Supports reasoning by providing specific conclusions. \\ \hline
    \end{tabular}
    \caption{Comparison of Learning, Reasoning, and Inference}
\end{table}
\subsection{Inference techniques in robotics}

    \subsubsection{Bayesian Inference}
    Bayesian inference is a probabilistic technique that updates the likelihood of a hypothesis based on prior knowledge and observed data. In robotics, it is widely used for sensor fusion, where data from multiple sensors like LiDAR, cameras, and IMUs are combined to estimate the robot's position or environment. This method handles uncertainties effectively, making it ideal for localization tasks such as Monte Carlo Localization. \cite{bishop-2006-pattern-recognition}
    
    \subsubsection{Maximum Likelihood Estimation (MLE)}
    Maximum Likelihood Estimation determines the parameters of a model that maximize the likelihood of observed data. In robotics, MLE is often used for model fitting, such as estimating the motion dynamics of a robot or calibrating sensor parameters. For example, a robot can use MLE to optimize its camera parameters to improve visual recognition accuracy. \cite{bishop-2006-pattern-recognition}
    
    \subsubsection{Monte Carlo Methods}
    Monte Carlo methods are stochastic approaches that use random sampling to estimate outcomes and probabilities. In robotics, Monte Carlo Localization is a prominent application, where particle filters estimate a robot’s pose using sensory data and motion models. These methods are particularly effective in high-dimensional and nonlinear systems, such as planning collision-free paths for drones. \cite{russell-2010-artificial-intelligence}
    
    \subsubsection{Deductive Inference}
    Deductive inference involves drawing specific conclusions from general rules or knowledge. Robots employ this approach for rule-based decision-making, where predefined if-then rules dictate actions based on observed states. For example, a robot might deduce that an obstacle detected ahead requires it to stop and plan an alternate path. This method is intuitive and interpretable. \cite{russell-2010-artificial-intelligence}
    
    \subsubsection{Neural Network-Based Inference}
    Neural network-based inference uses artificial neural networks to predict outcomes based on learned patterns in data. In robotics, this approach is widely applied in tasks like object detection using convolutional neural networks (CNNs) or motion prediction with recurrent neural networks (RNNs). For instance, a drone might infer the trajectory of a moving object to avoid collisions. Neural inference excels in handling complex, high-dimensional data. \cite{lecun-2015-deep-learning}
    
    \subsubsection{Fuzzy Inference}
    Fuzzy inference applies fuzzy logic to handle uncertainties and vagueness in data. Robots use this method for decision-making in imprecise or noisy environments. For example, a robotic vacuum might decide how much to adjust its path based on partially observable obstacles. Fuzzy inference provides intuitive and human-like reasoning capabilities, making it suitable for navigation and human-robot interaction. \cite{pfeifer-2007-embodied-intelligence}