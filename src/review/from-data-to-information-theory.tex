\chapter{From data to information theory}
In this chapter we  draw the boundries between data , sensory data, sensor and data fusion, information , etc
\section{Data}

\section{Information} Information is processed, structured, or organized data that has been given meaning and context, making it understandable and useful. In the following table you can find the difference between information and perception
        \begin{tabular}{|l|l|l|}
            \hline
            \textbf{Aspect} & \textbf{Perception} & \textbf{Information} \\ \hline
            \textbf{Definition} & The process of interpreting sensory input. & Processed data that provides meaning and context. \\ \hline
            \textbf{Process vs. Output} & A dynamic process involving sensors and interpretation. & A static or dynamic product resulting from processing. \\ \hline
            \textbf{Role} & A mechanism to generate understanding of the environment. & Provides insights or knowledge for decision-making. \\ \hline
            \textbf{Dependency} & Requires sensory input or stimuli. & Can exist independently (e.g., in a database or memory). \\ \hline
            \textbf{Examples in Humans} & Seeing a tree, hearing a sound. & Recognizing the tree as an oak or identifying the sound as music. \\ \hline
            \textbf{Examples in Robots} & Using cameras and algorithms to recognize objects. & Storing recognized objects as labeled data. \\ \hline
        \end{tabular}

\section{Knowledge} 
Refers to the collection of information, facts, concepts, or skills that an individual has acquired through learning, experience, or education. An example of knowledge in robotics or any other intelligent agent, a database of facts or pre-programmed rules (e.g., rules of chess) can be considered as knowledge. In humans, knowing that water boils at 100 degree is a knowledge. Knowledge is the static outcome of learning and experience; it is what we know. 
        \begin{itemize}
            \item Is on aspect of knowledge predicting future states from current and past states?
            \item What is the difference between knowledge and information?
        \end{itemize}

        
\section{Information theory}

\section{Signal processing}

\section{Questions}


