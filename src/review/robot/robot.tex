\chapter{Cognitive dynamic robots}
A robot is a programmable, multifunctional machine designed to perform tasks by sensing its environment, processing information, and taking action either autonomously or semi-autonomously. \cite{siciliano-2016-springer-handbook-robotics}

\section{The main components of a robot}:
The main components of a robot can be divided into five essential categories:

    \subsection{Physical components}
        \textbf{Sensors}: Allow robots to perceive both their environment and internal states, using sensors like cameras and LiDAR or IMUs.
        
        \textbf{Actuators}: Enable the robot to move or perform tasks by converting control signals into physical motion.
        Examples include motors, servos, robotic arms, and wheels.
    
        \textbf{Power Supply}: Provides the energy needed to operate the sensors, actuators, and processing unit.
        Examples include batteries, solar cells, and fuel cells.
    
        \textbf{Mechanical Structure}: The physical frame or body of the robot that supports and integrates the other components.
        Examples include wheels, legs, manipulators, and drone frames.

    \subsection{Virtual components:}
        \textbf{Controller (Processing Unit)}: The "brain" of the robot that processes sensor inputs, makes decisions, and sends commands to actuators.
        Examples include microcontrollers, processors, and embedded systems.

        \textbbf{Goals} A goal in robotics refers to a desired outcome or end state that a robot aims to achieve through its actions. Unlike tasks, goals are typically long-term and result-oriented, requiring the robot to plan, adapt, and choose appropriate tasks to reach the desired objective. Goals can be static or dynamic, depending on environmental changes or new information. In autonomous systems, goals often drive decision-making by determining which tasks to prioritize and how to allocate resources. In cognitive architectures, goal management is critical for enabling goal-directed behavior and learning from experience.
        \cite{scassellati-2001-goal-directed-behavior-in-robotics}
        
        \textbf{Tasks} A task in robotics refers to a specific action or set of actions that a robot performs to achieve a part of a process or solve a particular problem. Tasks are typically short-term, concrete, and action-oriented, involving actions like moving objects, navigating environments, or grasping items. Tasks are often predefined by a human operator or generated as part of a robot’s planning process. In multi-robot systems (MRS), tasks can be divided and distributed among different robots to achieve greater efficiency. Tasks are essential for executing broader goals and require task allocation mechanisms to optimize performance in dynamic environments.
        \cite{gerkey-2004-task-allocation-multi-robot-systems}

        


\section{Different robots}

    \subsection{Non-Intelligent Robots:} These robots operate based on pre-programmed instructions with limited or no adaptability to changing environments. Examples include automated assembly-line robots or simple vacuum robots. \cite{bekey-2005-autonomous-robots}
    
    \subsection{Intelligent Robots:} These robots are capable of perceiving their environment, reasoning, and making decisions autonomously, often using AI techniques like machine learning. Examples include humanoid robots, autonomous vehicles, or drones with adaptive navigation systems. \cite{siciliano-2016-springer-handbook-robotics}
    
        \textbf{Cognitive Robots:} A subset of intelligent robots, cognitive robots are designed to mimic human-like cognitive processes, such as reasoning, problem-solving, memory, and understanding intentions. They focus on emulating human intelligence and are often used for tasks requiring higher-order reasoning and natural interaction, such as social robots like Pepper. \cite{roy-2005-cognitive-robot}

        \textbf{Cognitive dynamic robots}
        
        \textbf{Concinnous robots}
        Probably never achievable. Pain must reduce computation resources in robot.They probably can never answer the question "What burning feels like" from recognized patterns in their in sensory data in their memory. Another aspect of pain might be messiness of words or meanings as sensory patterns in their memory. 



\section{Different goals}
\section{Feedback system}

\section{Control theory}

\section{Sensors}
\begin{itemize}
    \item proprioceptive and exteroceptive sensors?
    \item sensory data
    \item Are sensory data only for state estimation ?
\end{itemize}
\section{State estimation}
\subsection{Questions}
\begin{itemize}
    \item Is relating current sensor data with previous control inputs to solve problem a kind of state estimation?
\end{itemize}
\section{Signal processing}
\subsection{Questions:}
\begin{itemize}
    \item Are there pre-defined goals in intelligent agents?
    \item feedback system
    \item Should a robot have a model that relates current sensory data with control inputs?
\end{itemize}