\chapter{Language}
Either "Ferdinand de Saussure" or "Jacques Derrida" or "Edmund Husserl" say that "When I talk I understand the presence of what I am thinking about"

What is alphabet? "Edmund Husserl" says that "Sound is Awareness" Thomas Nagel has explored the concept of aural awareness in his philosophical discussions in his paper "What is it like to be a bat?"

    \section{Communication emergence in MRS}

    \section{Inner speech}
        \textbf{The importance of inner speech in self-awareness} \todo{Check this paper: \url{https://www.frontiersin.org/journals/robotics-and-ai/articles/10.3389/frobt.2020.00016/full}}

    \section{Language grounding with mutual presence and sensing the same concept or object at the same time}

    \section{Language grounding by comparing pattern in the mind}
    %this is just possible for machines in human it must be described

    \section{Concept composition}
        \todo{build corner, straight,  build left and associate turn left or straight left }

    \section{Formation and refinement}
        \textbf{Formation}

        \textbf{Refinement} Everytime an experience is is repeated the language must become more granular


    



\section{Todo}
    \todo{Language grounding methods in robots}
    \begin{itemize}
        \item Did first word existed or alphabet? I think word. 
        \item "time passage" is not a repeated pattern, then how can a word be corresponded to it?
        \item Should a self-aware robot talk to itself?
        \item Does talking to self needs another agent?
        \item Is abnormality detection over a continuous time interval a word or an alphabet?
        \item Can robot make sense of a word or a sentence without having experience it?
        \item When should a robot consider a set of ordered alphabets as a meaningful word? 
        \item When does a robot consider a set of ordered words as a sentence?
        \item Can a robot build meaningful language structures that it knows will never occur in the real world for any agent? words such as beautiful, ugly, god
        \item Maybe emotional words can be built by growth rate or first derivative.
        \item Does the alphabet have anything to do with different levels of derivative as growth rates?
        \item Is one aspect of language to build future probable meaningful structures?
        \item What is the definition of structure?
        \item Does a human, before saying something to another, say it to himself?
        \item Is keeping accidental composition of words or alphabets in mind to continuously assess their meaningfulness an aspect?
        \item what about two opposite words or sentences?
        \item Is there a knowledge in the reality in alphabets we don't have opposite or similarity concepts?
        \item Can neuro-symbolic AI be related to self-awareness? \href{https://en.wikipedia.org/wiki/Neuro-symbolic_AI}{url}
    \end{itemize}