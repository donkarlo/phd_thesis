\documentclass{book}

% ======================
% Packages (load after \documentclass)
% ======================
\usepackage{hyperref}
\usepackage{titlesec}
\usepackage[table]{xcolor}   % table row/col coloring
\usepackage{array}            % better tables
\usepackage[colorinlistoftodos]{todonotes}
\setcounter{tocdepth}{4}
\usepackage{amsmath}


% Graphics
\usepackage{tikz}
\definecolor{mainfill}{HTML}{FFF2CC}
\definecolor{mainborder}{HTML}{D6B656}
\tikzset{
    mainnode/.style={
        draw=mainborder,
        fill=mainfill,
        rectangle,
        align=center,
        inner sep=5pt,
        rounded corners,
        line width=0.8pt,
        text=black,
    }
}
%% fit is for process I think
\usetikzlibrary{graphs,graphdrawing,arrows.meta,mindmap,fit,positioning}
\usegdlibrary{layered,force}

%% For hierarchical trees
\usepackage[edges]{forest}

%% Enforce non-floating placement with [H]
\usepackage{float}
\usepackage{caption}
%% moves all figure captions to right
\captionsetup{singlelinecheck=false}


% Convenience macro for clickable file paths (you already had this)
\newcommand{\filepath}[1]{\href{file://#1}{\path{#1}}}

% This part is for another and better form code highlighting
\usepackage{minted}
\usemintedstyle{friendly}
%%setting for all syntaxes
\setminted{
    fontsize=\small,
    linenos,
    breaklines,
    frame=single,
    tabsize=4,
    autogobble,
}
%% if you want something special for python
\setminted[python]{fontsize=\footnotesize}
%% a short form for minted python inline
\newcommand{\pyl}[1]{\mintinline{python}{#1}}